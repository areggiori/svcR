\documentclass{article}
\usepackage[ae,hyper]{Rd}
\begin{document}
\HeaderA{iris\_att}{data matrix}{iris.Rul.att}
\keyword{datasets}{iris\_att}
\begin{Description}\relax
this data ...
\end{Description}
\begin{Usage}
\begin{verbatim}data(iris_att)\end{verbatim}
\end{Usage}
\begin{Format}\relax
a matrix of data
\end{Format}
\begin{Source}\relax
a matrix of data
\end{Source}
\begin{References}\relax
a matrix of data
\end{References}

\HeaderA{iris\_mat}{data matrix}{iris.Rul.mat}
\keyword{datasets}{iris\_mat}
\begin{Description}\relax
this data ...
\end{Description}
\begin{Usage}
\begin{verbatim}data(iris_mat)\end{verbatim}
\end{Usage}
\begin{Format}\relax
a matrix of data
\end{Format}
\begin{Source}\relax
a matrix of data
\end{Source}
\begin{References}\relax
a matrix of data
\end{References}

\HeaderA{iris\_var}{data matrix}{iris.Rul.var}
\keyword{datasets}{iris\_var}
\begin{Description}\relax
this data ...
\end{Description}
\begin{Usage}
\begin{verbatim}data(iris_var)\end{verbatim}
\end{Usage}
\begin{Format}\relax
a matrix of data
\end{Format}
\begin{Source}\relax
a matrix of data
\end{Source}
\begin{References}\relax
a matrix of data
\end{References}

\HeaderA{findModelCluster}{Computation of clustering model by support vector machine}{findModelCluster}
\keyword{cluster}{findModelCluster}
\begin{Description}\relax
SvcR implements a clustering algorithm based on separator search in a feature 
space between points described in a data space. Data format is defined by 
an attribute/value table (matrix). The data are transformed within a kernel 
to a feature space into a unic cluster bounded with a ball radius and support vectors. 
We can used the radius of this ball in the data space 
to reconstruct the boundary shaped now in several clusters.
\end{Description}
\begin{Usage}
\begin{verbatim}
findModelCluster(MetOpt="", MetLab="", KernChoice="", Nu="", q="", K="", G="", Cx="", Cy="", DName="", fileIn="")
\end{verbatim}
\end{Usage}
\begin{Arguments}
\begin{ldescription}
\item[\code{MetOpt}] option taking value 1 (randomization) or 2 (quadratic programming) 
\item[\code{MetLab}] option taking value 1 (grid labelling) or 2 (mst labelling) or 3 (knn labelling) 
\item[\code{KernChoice}] option taking value 0 (Euclidian) or 1 (RBF) or 2 (Exponential) 
\item[\code{Nu}] kernel parameter  
\item[\code{q}] kernel parameter 
\item[\code{K}] number of neigbours on the grid 
\item[\code{G}] size of the grid 
\item[\code{Cx}] 1st data coordinate to plot for 2D cluster extraction 
\item[\code{Cy}] 2nd data coordinate to plot for 2D cluster extraction 
\item[\code{DName}] Name of data which is the prefix of files :
\file{DName\_mat.txt}, 
\file{DName\_att.txt}, 
\file{DName\_var.txt} 
\item[\code{fileIn}] path where to find files 
\end{ldescription}
\end{Arguments}
\begin{Details}\relax
format of \file{DName\_mat.txt} (data matrix): 
1 1 5.1       
1 2 3.5
2 3 1.4
it mean mat[1, 1] = 5.1, mat[1, 2] = 3.5, mat[2, 3] = 1.4

format of \file{DName\_att.txt} : 
X1
X2
it mean X1 is the name of first column of the data matrix, X2 is the name of the second column of the data matrix

format of \file{DName\_var.txt} : 
v1
v2
it mean v1 is the name of first line of the data matrix, v2 is the name of the second line of the data matrix
\end{Details}
\begin{Value}
no return
\end{Value}
\begin{Author}\relax
Nicolas Turenne - INRA France \email{nicolas.turenne@jouy.inra.fr}
\end{Author}
\begin{References}\relax
N.Turenne , Some Heuristics to speed-up Support Vector Clustering , technical report 2006, INRA, France 
\url{http://migale.jouy.inra.fr/~turenne/svc.pdf}
\end{References}
\begin{Examples}
\begin{ExampleCode}

## exemple with iris data

MetOpt  = 1;    # optimisation method with randomization
MetLab  = 1;    # grid labelling
KChoice = 1;    # 0: eucli 1: radial 2: radial+dist 
Nu      = 0.7; 
q       = 1200;   # lot of clusters
K       = 1;    # only 1  nearest neighbour for clustering
Cx = Cy = 0; # we use principal component analysis factors
G       = 13; # size of the grid for cluster labelling
DName   = "iris";
fileIn  = ""; # fileIn migth be such as "D:/R/library/svc/", if NULL it will work on iris data

findModelCluster(MetOpt, MetLab, KChoice, Nu, q, K, G, Cx, Cy, DName, fileIn); 

\end{ExampleCode}
\end{Examples}

\end{document}
